\documentclass[titlepage]{article}

\usepackage{amsmath}

\title{Effective Vaccination Strategies for an H1N1 Epidemic in Ithaca, New York}
\author{Prepared by Paul Chesnais (pmc85), Christopher Silvia (cps232), \\ and Ryan Vogan (rcv39) in completion of Project 01 for MATH 3610}
\date{September 30, 2015}
\begin{document}
\maketitle

\section{Introduction}
	For this project, we were tasked with modeling a H1N1 outbreak in Ithaca, New York. The goal of this effort was to provide local government officials with a recommended plan for distributing 4000 vaccines among the city's approximately 60000 inhabitants. The strategies that officials are particularly interested in are:
	\begin{itemize}
		\item[1.]
			Focus vaccination efforts on more connected individuals
		\item[2.]
			Focus vaccination efforts on more frail or susceptible individuals
	\end{itemize}
	They wanted to know which strategy would be better under two possible scenarios:
	\begin{itemize}
		\item[1.]
			An outbreak of the H1N1 virus
		\item[2.]
			An outbreak of a new H1N1 strain that is twenty times as deadly
	\end{itemize}
	When evaluating possible strategies, we considered the following two metrics:
	\begin{itemize}
		\item[1.]
			Total number of deaths
		\item[2.]
			Total number of person-months of infection
	\end{itemize}
	Therefore, a particular distribution of vaccines is considered better if it saves more lives and prevents more suffering. 

\section{Deterministic Model}
\subsection{Algorithm}

\subsection{Assumptions and Parameter Justifications}
	//TODO
\subsection{Predictions}
	//TODO
\subsection{Validation and Robustness Testing}
	//TODO
\subsection{Strengths and Weaknesses}
	//TODO
\subsection{Future Work}
	//TODO

\section{Stochastic Model}
\subsection{Algorithm}
	//TODO
\subsection{Assumptions and Parameter Justifications}
	//TODO
\subsection{Predictions}
	//TODO
\subsection{Validation and Robustness Testing}
	//TODO
\subsection{Strengths and Weaknesses}
	//TODO
\subsection{Future Work}
	//TODO

\section{Conclusions}
	//TODO
\section{References}
	//TODO
\section{Source Code}
	//TODO
\section{Individual Contributions}
	//TODO

\end{document}